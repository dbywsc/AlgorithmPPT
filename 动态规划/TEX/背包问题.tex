\documentclass{beamer}

%---------- 主题设置 ----------
\usetheme{Berlin}               % 可选主题:Dresden, CambridgeUS, Malmoe…
\usecolortheme{orchid}          % 可选配色:beaver, orchid, seagull…
\setbeamertemplate{navigation symbols}{} % 隐藏导航图标

%---------- 中文字体配置 ----------
\usepackage[UTF8, fontset=mac]{ctex}

%---------- 常用宏包 ----------
\usepackage{graphicx}
\usepackage{listings}
\usepackage{tikz}
\usepackage{minted}
\usepackage{changepage}
\usepackage{graphbox}
\usepackage{twemojis}
\usepackage{relsize}
\newcommand{\myemoji}[1]{\raisebox{-.2ex}{\twemoji[height=1.25\fontcharht\font`X]{#1}}}
%---------- 自定义命令 ----------
% C++ 符号宏,依赖 hyperref 和 relsize
% 来自 GitHub 上的仓库 tcbrindle/wg21papertemplate 中的 ./common.tex 文件
\newcommand{\Cpp}{\texorpdfstring{C\kern-0.05em\protect\raisebox{.35ex}{\textsmaller[2]{+\kern-0.05em+}}}{C++}}
% 举例:让 \fdf 变成加粗显示
\newcommand{\fdf}[1]{\alert{\textbf{#1}}}
% 如果你只是想要一个空行,也可以用 \vspace{.5\baselineskip} 或者 \medskip
\newcommand{\emptyline}{\vspace{.5\baselineskip}}


% 用于数学公式中的自定义命令
\newcommand{\nequiv}{\not\equiv} % 不同余
\newcommand{\lcm}{\operatorname{lcm}} % 最小公倍数
\newcommand{\lpd}{\operatorname{lpd}} % 最小素因数

\title{背包问题}
\author{dbywsc}
\date{2025/8}
\begin{document}
\frame{\titlepage}
\begin{frame}
	\frametitle{目录}
	\tableofcontents
\end{frame}
\section{引入}
\begin{frame}
背包问题是线性动态规划中较为特殊的一种,并且变式较多,在算法竞赛中经常出现。
\end{frame}
\section{01背包问题}
\begin{frame}
P1048 [NOIP 2005 普及组] 采药 \\ 
给定一个容量为 $m$ 的背包,并且有 $n$ 件物品,每件物品有价值 $w_i$ 和重量 $v_i$ 。在\fdf{每件物品只能选一次}的情况下问如何选择物品能够在容量不超过 $m$ 的情况下价值最大。这样的问题就叫做\fdf{01背包}。
\end{frame}
\begin{frame}
先考虑设置二维状态:\\ 
设 $f_{i, j}$ 为选择第 $i$ 种物品,容量为 $j$ 时的最大价值,那么每次我们都可以从前一件物品中转移过来,所以状态转移方程为:
$$f_{i, j} = \max(f_{i - 1, j}, f_{i - 1, j - v_i} + w_i)$$
接下来考虑边界,显然有 $f_{0, j} = 0, j \in [0, m]$ ,最后的答案应该为 $f_{n, m}$ 。
由于需要枚举每一个物品,并且对于每一个物品要枚举全部的状态,所以这种做法的时空复杂度都为 $O(nm)$ 。
\end{frame}
\begin{frame}[fragile]
\begin{onlyenv}
\begin{minted}[fontsize = \scriptsize]{cpp}
void solve(void) {
    for(int i = 1; i <= n; i++) {
        for(int j = 0; j <= m; j++) {
            f[i][j] = f[i - 1][j];
            if(j >= v[i])
                f[i][j] = std::max(f[i][j], f[i - 1][j - v[i]] + w[i]);
        }
    }
    std::cout << f[n][m];
}
\end{minted}
\end{onlyenv}
这样做时空复杂度皆为 $O(nm)$ 。\\
对于朴素做法,时间复杂度已经无法优化,但是可以优化空间复杂度。
\end{frame}
\begin{frame}[fragile]
注意到每次转移状态时, $f_{i,j}$ 只会从 $f_{i - 1, j - v_i \ / \ j}$ 转移过来,也就是说每次只需要和前一个 $i$ 进行比较就可以了,所以考虑删去第一维 $i$ ,变成 $f_j = \max(f_j, f_{j - v_i} + w_i)$ 。然而,直接删掉第一维后维持原本的遍历顺序并不与二维状态等价:\\ 
\begin{onlyenv}
\begin{minted}[fontsize = \scriptsize]{cpp}
void solve(void) {
    for(int i = 1; i <= n; i++) {
        for(int j = v[i]; j <= m; j++) {
                f[j] = std::max(f[j], f[j - v[i]] + w[i]);
        }
    }
}
\end{minted}
\end{onlyenv}
在上面的代码中,$max(f_j, f_{j - v_i} + w_i)$ 其实是 $(i, j)$ 和 $(i, j - v_i)$ 进行比较,然而我们应该拿 $(i - 1, j - v_i)$ 进行比较。\\ 
要想解决也非常简单,只需要改变一下第二层的循环顺序就可以了。
\end{frame}
\begin{frame}[fragile]
\begin{onlyenv}
\begin{minted}[fontsize = \scriptsize]{cpp}
void solve(void) {
    for(int i = 1; i <= n; i++) {
        for(int j = m; j >= v[i]; j--) {
                f[j] = std::max(f[j], f[j - v[i]] + w[i]);
        }
    }
}
\end{minted}
\end{onlyenv}
因此,最终时间复杂度为 $O(nm)$ ,空间复杂度为 $O(m)$ 。
\end{frame}
\begin{frame}
下面来解释一下为什么可以这样做:\\ 
由于 $j - v_i$ 一定是小于等于 $j$ 的,如果顺序进行遍历, $f_{j - v_i}$ 一定会在 $f_j$ 前被更新,我们知道 $i - 1$ 层是会更新到第 $i$ 层的,所以等轮到 $f_j$ 的时候,我们会拿第 $i$ 层的 $f_j$ 更新它,因此会造成错误;如果倒序遍历,我们会先拿 $f_j$ 更新,由于 $j - v_i$ 排在它后面,所以此时还没有被更新到第 $i$ 层,因此我们其实是在拿 $i - 1$ 层的 $j - v_i$ 更新 $j$ ,就不会造成错误了。 \\ 
因此最后的答案就应该为 $f_m$ 。\\ 
随着算法竞赛的发展,一维的01背包状态转移方程已经成为了最基本的公式。绝大部分题目都会卡掉二维的做法,并且这个公式是其他背包变形的基础。
\end{frame}
\begin{frame}
由于本节内容较多,因此每个小节后都额外给出一些练习题:\\ 
P1049 装箱问题\\ 
P1060 开心的金明\\
P1734 最大约数和\\
P1510 精卫填海\\
P1466 [USACO2.2] 集合 Subset Sums
\end{frame}
\section{完全背包问题}
\begin{frame}
P1616 疯狂的采药 \\ 
给定一个容量为 $m$ 的背包,并且有 $n$ 件物品,每件物品有价值 $w_i$ 和重量 $v_i$ 。在\fdf{每件物品可以无限取}的情况下问如何选择物品能够在容量不超过 $m$ 的情况下价值最大。这样的问题就叫做\fdf{完全背包}。
\end{frame}
\begin{frame}
我们同样设 $f_{i, j}$ 为选第 $i$ 种物品,容量为 $j$ 时的最大价值,与01背包的选或不选的状态不同的是,由于物品是无限的,因此我们对与第 $i$ 件物品,可以选 $1, 2, 3, ..., k$ 件物品,显然选择的上界 $k$ 应该是 $\lfloor \frac{m}{v_i} \rfloor$ 。  \\ 
于是我们可以列出状态转移方程:
$$f_{i, j} = \max_{k = 1}^{\lfloor \frac{m}{v_i} \rfloor}(f_{i - 1, j}, f_{i - 1, j - k \times v_i} + k \times w_i)$$
\end{frame}
\begin{frame}[fragile]
\begin{onlyenv}
\begin{minted}[fontsize = \scriptsize]{cpp}
void solve(void) {
    for(int i = 1; i <= n; i++) {
        for(int j = 0; j <= m; j++) {
            f[i][j] = f[i - 1][j];
            for(int k = 1; k <= m / v[i]; k++) {
                if(k * v[i] <= j) 
                    f[i][j] = std::max(f[i][j], 
                        f[i - 1][j - k * v[i]] + k * w[i]);
            }
        }
    }
    std::cout << f[n][m];
}
\end{minted}
\end{onlyenv}
显然,这样做的空间复杂度是 $O(nm)$ ,时间复杂度是 $O(nm^2)$ 。\\
接下来考虑优化。
\end{frame}
\begin{frame}
先将朴素的状态转移方程拆开:
$$f_{i, j} = \max(\{f_{i - 1,j }, f_{i - 1, j - v_i} + w_i, f_{i - 1, j - 2v_i} + 2w_i, ..., f_{i - 1, j - kv_i} + kw_i\})$$ 
那么 $f_{i, j - v}$ 的状态就为:
$$f_{i, j - v_i} = \max(\{f_{i - 1, j - v_i}, f_{i - 1, j - 2v_i} + w_i, ..., f_{i - 1, j - v_i} + (k - 1)w_i\})$$ 
可以发现, $f_{i, j}$ 其实可以表示为 $f_{i, j} = \max(f_{i - 1, j}, f_{i, j - v_i} + w_i)$ 。它和01背包的状态转移方程及其相似。
\end{frame}
\begin{frame}[fragile]
因此我们可以先把时间复杂度优化到 $O(nm)$ :
\begin{onlyenv}
\begin{minted}[fontsize = \scriptsize]{cpp}
void solve(void) {
    for(int i = 1; i <= n; i++) {
        for(int j = 0; j <= m; j++) {
            f[i][j] = f[i - 1][j];
            if(j >= v[i]) f[i][j] = std::max(f[i][j], f[i][j - v[i]] + w[i]);
        }
    }
    std::cout << f[n][m];
}
\end{minted}
\end{onlyenv}
\end{frame}
\begin{frame}
现在来回忆一下01背包的状态转移方程:
$$f_{i, j} = \max(f_{i - 1, j}, f_{i - 1, j - w_i} + v_i)$$
在01背包问题中,由于 $f_i$ 永远只会从 $f_{i - 1}$ 层转移过来,所以我们可以用滚动数组的方式优化空间复杂度,将第一维删去,此时为了防止前面的项被提前更新,我们采用倒着遍历的方式。
而在完全背包问题中,我们发现,$f_i$ 只会从同一层转移过来:
$$f_{i, j} = \max(f_{i - 1, j}, f_{i, j - v_i} + w_i)$$
所以我们同样可以是用滚动数组。由于是从同层转移的,因此我们也不需要考虑有些项被提前更新的情况,因此从前往后遍历就可以了。
\end{frame}
\begin{frame}[fragile]
\begin{onlyenv}
\begin{minted}[fontsize = \scriptsize]{cpp}
void solve(void) {
    for(int i = 1; i <= n; i++) {
        for(int j = v[i]; j <= m; j++) {
                f[j] = std::max(f[j], f[j - v[i]] + w[i]);
        }
    }
}
\end{minted}
\end{onlyenv}
因此,最终时间复杂度为 $O(nm)$ ,空间复杂度为 $O(m)$ 。
\end{frame}
\begin{frame}
本小节习题:\\ 
P2722 [USACO3.1] 总分 Score Inflation\\ 
P1679 神奇的四次方数\\ 
P1832 A+B Problem(再升级)\\ 
P1853 投资的最大效益 \\
P2918 [USACO08NOV] Buying Hay S\\
\end{frame}
\section{多重背包问题}
\begin{frame}[fragile]
给定一个容量为 $m$ 的背包,并且有 $n$ 种物品,每种物品有 $s_i$ 件,并且有重量 $v_i$ 和价值 $w_i$ ,问如何选能在总容量不超过 $m$ 的情况下使物品的价值之和最大,这样的问题就叫做\fdf{多重背包}。 \\ 
由于存在数量的限制,所以我们可以把这个问题看作是一个有 $\sum_{i = 1}^{n}s_i$ 件物品的01背包问题,因此可以套用朴素的01背包的状态转移方程,只需要稍作修改即可:
\begin{onlyenv}
\begin{minted}[fontsize = \scriptsize]{cpp}
    for(int i = 1; i <= n; i++) {
        for(int j = 0; j <= m; j++) {
            for(int ss = 0; ss <= s[i] && ss * v[i] <= j; ss++) {
                f[i][j] = std::max(f[i][j], f[i - 1][j - v[i] * ss]
                 + w[i] * ss);
            }
        }
    }
\end{minted}
\end{onlyenv}
显然,这种做法的时间复杂度是 $O(nms)$ ,空间复杂度是 $O(nm)$ 。在一定的规模下,这种做法非常实用:\\ 
P2347 [NOIP 1996 提高组] 砝码称重
\end{frame}
\begin{frame}
但是我们已经知道,当数据规模增大时,这种做法在时间和空间上都会超出限制,所以考虑如何优化。\\ 
P1776 宝物筛选\\
众所周知, $2^0, 2^1, ..., 2^n$ 可以拼凑出 $1$ 到 $2^{n + 1} - 1$ 中的任何一个数。因此我们可以反过来,把一组 $s$ 个物品拆分成 $2^0$ 个、 $2^1$ 个,$2^2$ 个.... $2^k$ 个共 $\log s$ 组,由于我们一定能够使用这 $\log s$ 组物品凭凑出 $1$ 到 $s$ 中的任意一个数,因此直接对它们做一次01背包就可以了。顺带地,我们还可以使用滚动数组优化空间复杂度。
\end{frame}
\begin{frame}[fragile]
\begin{onlyenv}
\begin{minted}[fontsize = \scriptsize]{cpp}
    int cnt = 0;
    for(int i = 1; i <= n; i++) {
        int a, b, s; std::cin >> b >> a >> s;
        int k = 1;
        while(k <= s) {
            cnt++;
            v[cnt] = a * k;
            w[cnt] = b * k;
            s -= k;
            k *= 2;
        }
        if(s > 0) {
            cnt++;
            v[cnt] = a * s;
            w[cnt] = b * s;
        }
    }
    n = cnt;
    for(int i = 1; i <= n; i++) {
        for(int j = m; j >= v[i]; j--) {
            f[j] = std::max(f[j], f[j - v[i]] + w[i]);
        }
    }
\end{minted}
\end{onlyenv}
时间复杂度 $O(nm \log s)$ ,空间复杂度 $O(m)$ 。
\end{frame}
\begin{frame}
本小节习题:\\ 
P6771 [USACO05MAR]Space Elevator 太空电梯\\ 
\end{frame}
\section{分组背包问题}
\begin{frame}
P1757 通天之分组背包\\ 
给定一个容量为 $m$ 的背包,并且有 $n$ 种物品,每种物品有重量 $v_i$ 和价值 $w_i$ ,并且每件物品属于不同的组别中。问在每组物品中只能选一个点情况下,如何选能在总容量不超过 $m$ 的情况下使物品的价值之和最大,这样的问题就叫做\fdf{分组背包}。 \\
设 $f_{k, j}$ 为选前 $k$ 组物品,容量为 $j$ 时的最大价值,此时转移分两种情况:要么不选第 $k$ 组的物品,要么枚举第 $k$ 组中物品,选最大值,因此状态转移方程为:
$$f_{k, j} = \max(f_{k - 1, j}, \max_{i = 1}^{s}(f_{k - 1, j - v_i} + w_i))$$
本质上,分组背包是01背包的变形,所以我们同样使用滚动数组优化。
\end{frame}
\begin{frame}[fragile]
\begin{onlyenv}
\begin{minted}[fontsize = \scriptsize]{cpp}
int main(void) {
    int n, m; std::cin >> m >> n;
    std::vector<PII> s[N];
    std::vector<int> f(N, 0);
    int cnt = 0;
    for(int i = 1; i <= n; i++) {
        int a, b, c; std::cin >> a >> b >> c;
        s[c].emplace_back(a, b);
        cnt = std::max(cnt, c);
    }
    for(int i = 1; i <= cnt; i++) {
        auto last = f;
        for(auto ss : s[i]) {
            int v = ss.first, w = ss.second;
            for(int j = m; j >= v; j--) {
                f[j] = std::max(f[j], last[j - v] + w);
            }
        }
    }
    std::cout << f[m];
    return 0;
}
\end{minted}
\end{onlyenv}
时间复杂度 $O(nm)$ ,空间复杂度 $O(n + m)$ 。
\end{frame}
\section{混合背包问题}
\begin{frame}
给定一个容量为 $m$ 的背包,并且有 $n$ 种物品。物品分为三类,第一类物品只能用一次,第二种物品可以用无限次,第三种物品可以用 $s_i$ 次。问如何选能在总容量不超过 $m$ 的情况下使物品的价值之和最大。这样的问题就叫做\fdf{混合背包}。\\ 
可以发现,第一类物品其实就是01背包问题,第二类物品就是完全背包问题,第三类物品就是多重背包问题。所以想要解决这个问题非常的简单,只需要对每一种物品进行分情况讨论,针对不同的物品使用不同的状态转移方程就可以了。\\
\end{frame}
\begin{frame}[fragile]
\begin{onlyenv}
\begin{minted}[fontsize = \scriptsize]{cpp}
void solve(void) {
    int n, m; std::cin >> n >> m;
    for(int i = 1; i <= n; i++) {
        int v, w, s;
        std::cin >> v >> w >> s;
        if(s == 0) {    //完全背包
            for(int j = v; j <= m; j++)
                f[j] = std::max(f[j], f[j - v] + w);
        } else {   //多重背包、01背包都当作01背包处理。
            if(s == -1) s = 1;
            for(int k = 1; k <= s; k *= 2) {
                for(int j = m; j >= k * v; j--) 
                    f[j] = std::max(f[j], f[j - k * v] + k * w);
                s -= k;
            }
            if(s) for(int j = m; j >= s * v; j--)
                    f[j] = std::max(f[j], f[j - s * v] + s * w);
        }
    }
    std::cout << f[m];
}
\end{minted}
\end{onlyenv}
练习题:P1833 樱花
\end{frame}
\section{有依赖的背包问题}
\begin{frame}
P1064 金明的预算方案\\ 
在这道题中,物品分为主件和附件,如果要买一个附件的话就必须要买它的主件。但是反过来说,买了主件不必购买全部的附件。在这种条件下,要求我们不超过背包容量的同时最大化价值。\\ 
我们可以把一个主件和他的全部附件归属于一个物品组中,此时就可以把这个问题当作分组背包处理。
\end{frame}
\begin{frame}[fragile]
\begin{onlyenv}
\begin{minted}[fontsize = \scriptsize]{cpp}
void solve(void) {
    int n, m; std::cin >> m >> n;
    for(int i = 1; i <= n; i++) {
        int v, w, q; std::cin >> v >> w >> q;
        if(!q) zhujian[i] = {v, v * w};
        else fujian[q].push_back({v, v * w});
    }
    for(int i = 1; i <= n; i++) {
        if(zhujian[i].first) {
            for(int j = m; j >= 0; j--) {
                for(int k = 0; k < (1 << fujian[i].size()); k++) {
                    int v = zhujian[i].first, w = zhujian[i].second;
                    for(int u = 0; u < fujian[i].size(); u++)
                        if(k >> u & 1) {
                            v += fujian[i][u].first;
                            w += fujian[i][u].second;
                        }
                    if(j >= v) f[j] = std::max(f[j], f[j - v] + w);
                }
            }
        }
    }
    std::cout << f[m];
}
\end{minted}
\end{onlyenv}
\end{frame}
\section{习题}
\begin{frame}
相对于每小节的练习题,此处的题目更有挑战性\\ 
P2370 yyy2015c01 的 U 盘\\ 
P4141 消失之物\\
P1156 垃圾陷阱\\
P3985 不开心的金明\\
P1455 搭配购买\\
P2170 选学霸\\
P1858 多人背包\\
P5662 纪念品\\ 
P5020 [NOIP2018 提高组] 货币系统\\
P1941 飞扬的小鸟\\
P5365 [SNOI2017]英雄联盟 \\
P2851 [USACO06DEC]The Fewest Coins G\\
P5322 [BJOI2019]排兵布阵\\ 
P1782 旅行商的背包\\
P2904 [USACO08MAR]River Crossing S\\
\end{frame}
\end{document}