\documentclass{beamer}

%---------- 主题设置 ----------
\usetheme{Berlin}               % 可选主题:Dresden, CambridgeUS, Malmoe…
\usecolortheme{orchid}          % 可选配色:beaver, orchid, seagull…
\setbeamertemplate{navigation symbols}{} % 隐藏导航图标

%---------- 中文字体配置 ----------
\usepackage[UTF8, fontset=mac]{ctex}

%---------- 常用宏包 ----------
\usepackage{graphicx}
\usepackage{listings}
\usepackage{tikz}
\usepackage{minted}
\usepackage{changepage}
\usepackage{graphbox}
\usepackage{twemojis}
\usepackage{relsize}
\newcommand{\myemoji}[1]{\raisebox{-.2ex}{\twemoji[height=1.25\fontcharht\font`X]{#1}}}
%---------- 自定义命令 ----------
% C++ 符号宏,依赖 hyperref 和 relsize
% 来自 GitHub 上的仓库 tcbrindle/wg21papertemplate 中的 ./common.tex 文件
\newcommand{\Cpp}{\texorpdfstring{C\kern-0.05em\protect\raisebox{.35ex}{\textsmaller[2]{+\kern-0.05em+}}}{C++}}
% 举例:让 \fdf 变成加粗显示
\newcommand{\fdf}[1]{\alert{\textbf{#1}}}
% 如果你只是想要一个空行,也可以用 \vspace{.5\baselineskip} 或者 \medskip
\newcommand{\emptyline}{\vspace{.5\baselineskip}}


% 用于数学公式中的自定义命令
\newcommand{\nequiv}{\not\equiv} % 不同余
\newcommand{\lcm}{\operatorname{lcm}} % 最小公倍数
\newcommand{\lpd}{\operatorname{lpd}} % 最小素因数

\title{双指针、单调栈、单调队列}
\author{dbywsc}
\date{2025/7}
\begin{document}
\frame{\titlepage}
\begin{frame}
	\frametitle{目录}
	\tableofcontents
\end{frame}
\section{双指针}
\begin{frame}
\frametitle{介绍}
\fdf{双指针}(two-pointer) 是一种简单而又灵活的技巧和思想,单独使用可以用来解决一些特定的问题,也可以配合其他算法实现不同的功能。\\ 
顾名思义,双指针即遍历时同时使用两个指针指向不同的位置,通过移动两个指针来维护不同的信息。双指针通常有以下两种形式:\\ 
\fdf{快慢指针},即两个指针从一个方向出发,但是移动速度不同。\\ 
\fdf{首尾指针},即两个指针从两端出发,向中间靠拢。
\end{frame}
\begin{frame}
\frametitle{快慢指针 -- I}
P1102 A-B 数对 \\ 
先将数组排序。由于排序后相同的数字一定是连续的,所以我们可以维护快慢指针 $r_1$ 和 $r_2$, $r_1$ 找最后一个 $a_{r_1} - a_i \leq c$ 的下标, $r_2$ 找最后一个 $a_{r_2} - a_i < c$ 的下标,那么中间的 $r_1 - r_2$ 个数一定是 $a_x - a_i = c$ 的个数,因此 $ans \ += r_1 - r_2$ 。
\end{frame}
\begin{frame}[fragile]
\frametitle{快慢指针 -- II}
\begin{onlyenv}
\begin{minted}[fontsize = \scriptsize]{cpp}
void solve(void) {
    int n, c; std::cin >> n >> c;
    i64 ans = 0;
    std::vector<int> a(n);
    for(int i = 0; i < n; i++) std::cin >> a[i];
    std::sort(a.begin(), a.end());
    int r1 = 0, r2 = 0;
    for(int i = 0; i < n; i++) {
        while(r1 < n && a[r1] - a[i] <= c) r1++;
        while(r2 < n && a[r2] - a[i] < c) r2++;
        ans += r1 - r2;
    }
    std::cout << ans << std::endl;
}
\end{minted}
\end{onlyenv}
由于在循环中, $r_1$ 和 $r_2$ 自始至终都是递增的,也就是说总共只会执行 $n$ 次,因此上面代码双指针部分的时间复杂度是 $O(n)$ 。
\end{frame}
\begin{frame}
\frametitle{首尾指针 -- I}
P8708 [蓝桥杯 2020 省 A1] 整数小拼接 \\ 
先对数组排序,然后维护首尾指针 $l$ 和 $r$ ,此时我们发现:\\ 
1.如果 $a_l + a_r < k$ \footnote{此处的 $+$ 指的是字符串拼接},那么 $l$ 到 $r$ 之间的所有数字和 $a_l$ 拼接的结果也必然\fdf{小于} $k$ ,因此 $ans$ 可以直接加上 $r - l$,之后移动 $l$ 到下一位。 \\ 
2.如果 $a_l + a_r = k$ ,那么 $l$ 到 $r$ 之间的所有数字和 $a_l$ 拼接的结果也必然\fdf{小于等于} $k$ ,因此 $ans$ 可以直接加上 $r - l$,之后移动 $l$ 到下一位,移动 $r$ 到前一位。\\
3.如果 $a_l + a_r > k$ ,那么 $l$ 到 $r$ 之间的所有数字和 $a_l$ 拼接的结果也必然\fdf{大于} $k$ ,没有合适的答案,所以移动 $r$ 到前一位。 \\ 
需要注意的是由于 $sort()$ 默认对字符串的排序使用的是 \fdf{字典序} ,而在本题中,字符串长度的优先级大于字典序的优先级,因此我们要自定义排序规则。
\end{frame}
\begin{frame}
\frametitle{首尾指针 -- II}
另外,由于本题中 $a_l + a_r$ 和 $a_r + a_l$ 是两种不同的方案,因此我们需要跑两次双指针,一次判断 $a_l + a_r$ ,一次判断 $a_r + a_l$ 。 \\ 
由于指针自始至终是单调的,所以本题的时间复杂度仍然是线性的。
\end{frame}
\begin{frame}[fragile]
\frametitle{首尾指针 -- III}
\begin{onlyenv}
\begin{minted}[fontsize = \scriptsize]{cpp}
bool cmp(std::string x, std::string y) {
    if(x.size() == y.size()) return x < y;
    return x.size() < y.size();
}
void solve(void) {
    int n; i64 k, ans = 0; std::cin >> n >> k;
    std::vector<std::string> a(n);
    for(int i = 0; i < n; i++) std::cin >> a[i];
    std::sort(a.begin(), a.end(), cmp);
    int l = 0, r = n - 1;
    while(l <= r)
        if(std::stol(a[l] + a[r]) < k) ans += r - l, l++;
        else if(std::stol(a[l] + a[r]) == k)  ans += r - l, l++, r--;
        else r--;
    l = 0, r = n - 1;
    while(l <= r) 
        if(std::stol(a[r] + a[l]) < k) ans += r - l, l++;
        else if(std::stol(a[r] + a[l]) == k) ans += r - l, l++, r--;
        else r--;
    std::cout << ans << std::endl;
}
\end{minted}
\end{onlyenv}
\end{frame}
\section{单调栈}
\begin{frame}
\frametitle{介绍}
\fdf{单调栈}是维护栈内元素单调性的数据结构。简单来说,如果新入栈的元素会让栈内元素不单调,就会不断的出栈,直到剩下的元素和新入栈的元素仍然满足单调性。\\ 
一般来说,单调栈存储的是当前元素在数组中的\fdf{下标} 。
\end{frame}
\begin{frame}
\frametitle{原理}
P5788 【模板】单调栈\\
显然,本题要求我们倒着维护一个\fdf{严格递减}单调栈。\\
实现起来其实非常简单,我们用数组模拟,也可以直接用 $stack$ 容器。当新元素入栈时,如果栈内为空,那么我们直接入栈,如果栈内不为空,那么先检查栈顶是否\fdf{小于等于}要插入的元素,如果小于说明不是严格递减的,所以出栈。之后重复执行这一步,直到栈顶不小于等于当前元素或者栈空,再将当前元素入栈。
\end{frame}
\begin{frame}[fragile]
\frametitle{实现}
\begin{onlyenv}
\begin{minted}[fontsize = \scriptsize]{cpp}
void solve(void) {
    std::stack<int> s;
    int n; std::cin >> n;
    std::vector<int> f(n + 1), a(n + 1);
    for(int i = 1; i <= n; i++) std::cin >> a[i]; 
    for(int i = n; i >= 1; i--) {
        while(s.size() && a[s.top()] <= a[i]) s.pop();
        if(s.size()) f[i] = s.top();
        else f[i] = 0;
        s.push(i);
    }
    for(int i = 1; i <= n; i++) std::cout << f[i] << " ";
}
\end{minted}
\end{onlyenv}
\end{frame}
\section{单调队列}
\begin{frame}
\frametitle{介绍}
与单调栈类似,\fdf{单调队列}是维护队内元素单调性的数据结构,通常来说队列内存储的依然是元素在数组中的下标。\\ 
不同的是当队内元素不单调时,我们既可以选择从\fdf{对首}出队,也可以选择从\fdf{队尾}出队。\\ 
因此,出了数组手动模拟之外,如果想要通过STL实现这一数据结构,则应该使用	 $deque$ (双端队列)。\\
双端队列有以下迭代器:\\ 
$dq.front()$ 返回队首。\\ 
$dq.back()$ 返回队尾。 \\ 
$dq.push_front() \ dq.push_back()$ 在对首、队尾入队。 \\ 
$dq.pop_front() \ dq.pop_back()$ 在对首、队尾出队。 \\ 
$dq.clear()$ 清空队列。
\end{frame}
\begin{frame}
\frametitle{原理}
P1886 滑动窗口 /【模板】单调队列\\
形式化的说,本题要求我们维护一段长度为 $k$ 的\fdf{连续}的区间,并且求出它的最值。\\
对于求区间最大值和最小值,我们可以分别跑两次单调队列,第一次维护\fdf{不增}单调队列,如果队尾\fdf{大于}当前元素,就从队尾出队,知道队列为空或者满足不增的单调性,再从队尾入队。如果当前已经处理到了第 $k$ 个以后的元素,那么每次判断新的元素时应该从队首把属于上个区间的元素出队(这个操作应该在维护单调性之前)。之后如果我们已经处理了 $k$ 个以上的元素,此时每次判断新元素就像相当于维护新的区间,因此每次都要输出当前的最值。\\ 
第二次跑一次\fdf{不减}单调队列,操作和第一次一样,不同点是如果如果队尾\fdf{小于}当前元素才执行出队操作。
\end{frame}
\begin{frame}[fragile]
\frametitle{实现}
\begin{onlyenv}
\begin{minted}[fontsize = \scriptsize]{cpp}
void solve(void) {
    int n, k; std::cin >> n >> k;
    std::deque<int> dq;
    std::vector<int> a(n + 1);
    for(int i = 1; i <= n; i++) std::cin >> a[i];
    for(int i = 1; i <= n; i++) {
        while(dq.size() && dq.front() + k <= i) dq.pop_front();
        while(dq.size() && a[dq.back()] > a[i]) dq.pop_back();
        dq.push_back(i);
        if(i >= k) std::cout<< a[dq.front()] << " ";
    } std::cout << std::endl;
    dq.clear();
    for(int i = 1; i <= n; i++) {
        while(dq.size() && dq.front() + k <= i) dq.pop_front();
        while(dq.size() && a[dq.back()] < a[i]) dq.pop_back();
        dq.push_back(i);
        if(i >= k) std::cout<< a[dq.front()] << " ";
    }
}
\end{minted}
\end{onlyenv}
\end{frame}
\section{习题}
\begin{frame}
\frametitle{习题}
本节学习的内容通常作为“优化技巧”出现在程序设计中,在算法竞赛中需要和一定的思考相结合,是非常实用的技巧。\\
P5745 【深基附B例】区间最大和 \\ 
P1638 逛画展 \\
CF2112C Coloring Game \\
P1901 发射站 \\
P2866 [USACO06NOV] Bad Hair Day S \\ 
P2032 扫描\\
P2698 [USACO12MAR] Flowerpot S \\
\end{frame}
\end{document}