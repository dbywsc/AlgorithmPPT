\documentclass{beamer}

%---------- 主题设置 ----------
\usetheme{Berlin}               % 可选主题:Dresden, CambridgeUS, Malmoe…
\usecolortheme{orchid}          % 可选配色:beaver, orchid, seagull…
\setbeamertemplate{navigation symbols}{} % 隐藏导航图标

%---------- 中文字体配置 ----------
\usepackage[UTF8, fontset=mac]{ctex}

%---------- 常用宏包 ----------
\usepackage{graphicx}
\usepackage{listings}
\usepackage{tikz}
\usepackage{minted}
\usepackage{changepage}
\usepackage{graphbox}
\usepackage{twemojis}
\usepackage{relsize}
\newcommand{\myemoji}[1]{\raisebox{-.2ex}{\twemoji[height=1.25\fontcharht\font`X]{#1}}}
%---------- 自定义命令 ----------
% C++ 符号宏,依赖 hyperref 和 relsize
% 来自 GitHub 上的仓库 tcbrindle/wg21papertemplate 中的 ./common.tex 文件
\newcommand{\Cpp}{\texorpdfstring{C\kern-0.05em\protect\raisebox{.35ex}{\textsmaller[2]{+\kern-0.05em+}}}{C++}}
% 举例:让 \fdf 变成加粗显示
\newcommand{\fdf}[1]{\alert{\textbf{#1}}}
% 如果你只是想要一个空行,也可以用 \vspace{.5\baselineskip} 或者 \medskip
\newcommand{\emptyline}{\vspace{.5\baselineskip}}


% 用于数学公式中的自定义命令
\newcommand{\nequiv}{\not\equiv} % 不同余
\newcommand{\lcm}{\operatorname{lcm}} % 最小公倍数
\newcommand{\lpd}{\operatorname{lpd}} % 最小素因数

\title{递归时间复杂度分析}
\author{dbywsc}
\date{2025/8}
\begin{document}
\frame{\titlepage}
\begin{frame}
	\frametitle{目录}
	\tableofcontents
\end{frame}
\section{一些符号}
\begin{frame}[fragile]
在之前的学习中,我们已经学会了使用大 $O$ 符号分析时间复杂度,比如:
\begin{onlyenv}
\begin{minted}[fontsize = \scriptsize]{cpp}
    for(int i = 1; i <= n; i++) {
        for(int j = 1; j <= m; j++) {
            ans += a[i][j];
        }
    }
\end{minted}
\end{onlyenv}
在上面的代码中,ans += a[i][j] 会执行 $n \times m$ 次,因此我们称其时间复杂度为 $O(nm)$ 。
当一段代码执行的次数类似于 $an^3 + bn^2 + cn + T$ 时,我们将其时间复杂度记做 $O(n^3)$ 。
事实上,我们使用的符号 $O$ 在复杂度分析中称之为 \fdf{大O表示法} (big-oh),这种忽略低次项、常数项和系数的时间复杂度叫 \fdf{渐进时间复杂度} 。
\end{frame}
\begin{frame}
除了big-oh之外,时间复杂度还有其他的表示符号。\\
设一段代码执行的严格时间为 $T(n)$ ,则有以下符号:\\
$\Theta(n)$ :读作 theta,用这种符号最精确的时间复杂度,即 $\Theta(n) = T(n)$ 。 \\ 
$O(n)$ :读作 big-oh,使用这种方法表示“小于等于”,即 $O(n) \leq T(n)$ 。 \\ 
$o(n)$ :读作 small-oh,使用这种方法表示 “小于”,即 $o(n) < T(n)$ 。 \\ 
$\Omega(n)$ : 读作 big-omega,使用这种方法表示 “大于等于”,即 $\Omega(n) \geq T(n)$ 。 \\ 
$\omega(n)$ : 读作 small-omega,使用这种方法表示 “大于”, 即 $\omega(n) > T(n)$ 。
在实际做题中,我们常用 $O(n)$ 估计时间;而在理论分析中,我们常用 $\Theta(n)$ 和 $O(n)$ 两种符号。
\end{frame}
\section{对数的基本运算}
\begin{frame}
在讲解递归时间复杂度之前,我们需要先熟悉一下对数的使用。
设 $a^b = N$ ,那么我们可以把这个式子写成 $\log_{a}N = b$ 。
可以发现,形如 $f(x) = \log_a x$ 与 $g(x) = a^x$ ,当二者的 $a$ 相等时 $f(x)$ 和 $g(x)$ 是互逆的,我们称 $f(x)$ 和 $g(x)$ 互为 \fdf{反函数} 。
下面列出几个常用的对数:\\
以 $10$ 为底的对数称之为 “常用对数”,记做 $\lg x$ 。\\
以 $e$\footnote{即欧拉常数,约等于 $2.71828182846$} 的对数称之为自然对数,记做 $\ln x$。\\
以 $2$ 为底的对数在算法竞赛中也很常用,\fdf{本节中简写为} $\log x$ 。
\end{frame}
\begin{frame}
下面是一些对数运算的法则:\\
$\log_a x + \log_a y = \log_a (xy)$  \\ 
$\log_a x - \log_a y = \log_ (\frac{x}{y})$ \\ 
$\log_a x^y = y \log_a x$ \\ 
$a^{\log_a b} = b$  \\ 
\end{frame}
\section{主定理}
\begin{frame}
在分析递归程序的时间复杂度时,常使用\fdf{主定理}(Master Theorem) 。 \\ 
设一个递归程序执行的时间为 $T(n)$ ,我们在递归的过程中,讲这个问题分治为了 $a$ 个规模为 $\frac{n}{b}$ 的子问题,每次递归额外带来的计算为 $f(n)$ ,即:
$$T(n) = a T(\frac{n}{b}) + f(n)$$
那么存在以下关系\footnote{严格来说本节主定理描述其实并不准确。在第二种情况和第三种情况中,遇到临界情况时需要额外判断。本节忽略了这一点,但不影响做初赛的题目。}:
$$T(n)  =  \left\{\begin{array}{ll}
\Theta(n^{\log_b a}) & f(n)  =  O(n^{\log_b (a) - \epsilon}) \ \  \epsilon > 0 \\
\Theta(f(n)) & f(n) = \Omega(n^{\log_b (a) + \epsilon}) \ \ \epsilon > 0 \\
\Theta(n^{\log_b a} \log ^{k + 1} n) & f(n) = \Theta(n^{\log_b a}\log^k n) \ \ k \geq 0
\end{array}\right.$$
其中, $\epsilon$ 是常数。
\end{frame}
\begin{frame}
第一种情况:当 $f(n) = O(n^{\log_b(a) - \epsilon})$ 时, $T(n) = \Theta(n^log_b a)$ 。 \\ 
e.g
$$T(n) = 2T(\frac{n}{2}) + \sqrt(n)$$
$a = 2, b = 2, log_b a = 1, f(n) = n^{\frac{1}{2}}$ \\
$f(n) = n^{\frac{1}{2}} = n^{\log_b a - \frac{1}{2}}$ 。取 $\epsilon = \frac{1}{2}$ ,可得 $T(n) = \Theta(n)$
\end{frame}
\begin{frame}
第二种情况:当 $f(n) = \Omega(\log_b(a) + epsilon)$ 时, $T(n) = \Theta(f(n))$ 。 \\ 
e.g
$$T(n) = 2T(\frac{n}{2}) + n^2$$
$a = b = 2, \log_b a = 1, f(n) = n^2$ \\ 
$f(n) = n^2 = n^{\log_b a + 1}$ ,取 $\epsilon = 1$ ,可得 $T(n) = \Theta(n^2)$
\end{frame}
\begin{frame}
第三种情况:当 $f(n) = \Theta(n^{\log_b a} \log^k n)$ 时, $T(n) = \Theta(n{\log_b a} \log^{k + 1} n))$ 。\\ 
e.g
$$T(n) = 2T(\frac{n}{2}) + n$$
$$a = b = 2, \log_b a = 1, f(n) = n$$ \\ 
当 $k = 0$ 时: $f(n) = n = n^{\log_b a} \times \log^0n$ ,可得 $T(n) = \Theta(n^{\log_b a} \log^{0 + 1} n) = n\log n$ 。
\end{frame}
\section{初赛题目选讲}
\begin{frame}
NOIP 2016 提高组 选择题第14题:\\ 
假设某算法的计算时间表达式为递推关系式:
$$T(n) = 2T(\frac{n}{4}) + \sqrt(n)$$
$$T(1) = 1$$
求算法时间复杂度。 \\ 
由主定理得, $a = 2, b = 4, \log_b a = \frac{1}{2}, f(n) = n^{\frac{1}{2}}$ 。 \\
根据第三种情况,取 $k = 0$ 时, $f(n) = n^{\log_b a} \times \log^0 n$ ,因此 $T(n) = \Theta(n^{\log_b a} \log^{0 + 1}n) = \Theta(n^{\frac{1}{2}} \log n)$\footnote{在初赛的选择题中,选项有时用big-oh表示法,此时如果同时存在big-oh和Theta取更精确的Theta,如果只有big-oh取big-oh即可} 。 
\end{frame}
\begin{frame}
NOIP 2015 提高组 选择题第十题:\\ 
假设某算法的计算时间表达式为递推关系式:\\ 
$$T(n) = T(n - 1) + n \ (n \in \mathbb{N}^*)$$
$$T(0) = 1$$
求算法时间复杂度。
这个问题不需要主定理,直接计算即可。
$$T(n) = T(n - 1) + n = T(n - 2) + (n - 1) + n$$
$$= T(n - 3) + (n - 2) + (n - 1) + n = ... $$
$$= T(0) + 1 + 2 + ... + (n - 1) + n$$
$$= 1 + \frac{n(n + 1)}{2}$$
如果取Theta,则 $T(n) = \Theta(1 + \frac{n(n + 1)}{2})$,如果取 big-oh ,那么渐进时间复杂度为 $O(n^2)$ 。
\end{frame}
\begin{frame}[fragile]
NOIP 2013 提高组 选择题第七题:
斐波那契数列的定义如下:
$$F_1 = F_2 = 1, F_n = F_{n - 2} + F_{n - 1} / (n \geq 3)$$
问用下面方式计算 $F_n$ 的时间复杂度是多少:
\begin{onlyenv}
\begin{minted}[fontsize = \scriptsize]{cpp}
int F(int n) {
    if(n <= 2) return 1;
    return F(n - 1) + F(n - 2);
}
\end{minted}
\end{onlyenv}
A. $O(1)$ , B. $O(n)$ ,C. $O(n^2)$ D. $O(F_n)$ \\
本题仅仅要求我们求出渐进时间复杂度,显然,计算任何一项的复杂度都不会超过 $F_n$ ,因此复杂度最大的项就是计算 $F_n$ 的复杂度,因此选D。
\end{frame}
\end{document}