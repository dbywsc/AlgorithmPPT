\documentclass{beamer}

%---------- 主题设置 ----------
\usetheme{Berlin}               % 可选主题:Dresden, CambridgeUS, Malmoe…
\usecolortheme{orchid}          % 可选配色:beaver, orchid, seagull…
\setbeamertemplate{navigation symbols}{} % 隐藏导航图标

%---------- 中文字体配置 ----------
\usepackage[UTF8, fontset=mac]{ctex}

%---------- 常用宏包 ----------
\usepackage{graphicx}
\usepackage{listings}
\usepackage{tikz}
\usepackage{minted}
\usepackage{changepage}
\usepackage{graphbox}
\usepackage{twemojis}
\usepackage{relsize}
\newcommand{\myemoji}[1]{\raisebox{-.2ex}{\twemoji[height=1.25\fontcharht\font`X]{#1}}}
%---------- 自定义命令 ----------
% C++ 符号宏,依赖 hyperref 和 relsize
% 来自 GitHub 上的仓库 tcbrindle/wg21papertemplate 中的 ./common.tex 文件
\newcommand{\Cpp}{\texorpdfstring{C\kern-0.05em\protect\raisebox{.35ex}{\textsmaller[2]{+\kern-0.05em+}}}{C++}}
% 举例:让 \fdf 变成加粗显示
\newcommand{\fdf}[1]{\alert{\textbf{#1}}}
% 如果你只是想要一个空行,也可以用 \vspace{.5\baselineskip} 或者 \medskip
\newcommand{\emptyline}{\vspace{.5\baselineskip}}


% 用于数学公式中的自定义命令
\newcommand{\nequiv}{\not\equiv} % 不同余
\newcommand{\lcm}{\operatorname{lcm}} % 最小公倍数
\newcommand{\lpd}{\operatorname{lpd}} % 最小素因数

\title{线性表}
\author{dbywsc}
\date{2025/7}
\begin{document}
\frame{\titlepage}
\begin{frame}
	\frametitle{目录}
	\tableofcontents
\end{frame}
\section{数组}
\begin{frame}
\frametitle{静态数组}
事实上,\fdf{数组}(Array)是一种线性的数据结构。它的特点是每个元素都是紧密相连着的,是一块连续的内存。\\ 
我们一般说的数组是\fdf{静态数组},也就是大小不可修改,形如 $int \ a[];$ 这样的数组。 \\ 
数组支持以 $O(1)$ 的时间复杂度随机访问元素,以 $O(1)$ 的时间复杂度在末尾插入元素。由于内存是连续的,所以无法直接在数组的中间或者头部插入或者一个元素。
\end{frame}
\begin{frame}[fragile]
\frametitle{动态数组 -- I}
在 C++ 中,我们通常将 $vector$ 容器作为动态数组使用。\\ 
$vector$ 既可以提前分配大小,也可以随着元素的增多动态的分配内存,下面是一些 $vector$ 的常用操作:\\
$std::vector<type> name(size)$ 创建一个 type 类型、名为 name 并且大小为 size 的 vector。\\
$std::vector<type> name$ 创建一个 type 类型,名为 name 的 vector 。 \\
$a.size()$ 可以用 size() \fdf{迭代器} 得到 vector 当前的容量。\\
\begin{onlyenv}
\begin{minted}[fontsize = \scriptsize]{cpp}
for(int i = 0; i < a.size(); i++) std::cout << a[i] << " ";
\end{minted}
\end{onlyenv}
可以像使用数组一样遍历和访问 vector 
\end{frame}
\begin{frame}[fragile]
\frametitle{动态数组 -- II}
可以使用 $begin()$ 和 $end()$ 迭代器表示 vector 中的收尾位置。\\
\begin{onlyenv}
\begin{minted}[fontsize = \scriptsize]{cpp}
sort(a.begin(), a.end());
\end{minted}
\end{onlyenv}
上面的例子即对一个 vector $a$ 做了从头到尾的升序排序。\\
$a.push_back(x)$ 以 $O(1)$ 的时间复杂度在 vector 末尾插入一个元素 $x$ \\ 
$a.insert(it, x)$ 以 $O(n)$ 的时间复杂度在 it 处插入一个元素 $x$ \\ 
$a.clear()$ 以 $O(n)$ 的时间复杂度清空 vector \\ 
$a.pop_back()$ 以 $O(1)$ 的时间复杂度删除数组末尾的元素 \\ 
$a.erase(it)$ 以 $O(n)$ 的时间复杂度删除 $it$ 处的元素,并且保证其余的元素仍然是紧密相连着的。
\end{frame}
\section{栈}
\begin{frame}
\frametitle{介绍}
\fdf{栈} (Stack) 是一种\fdf{后进先出}的数据结构。\\ 
栈通常能够实现以下操作:\\ 
$s.top()$ 访问\fdf{栈顶}元素\\
$s.push(x)$ 将 $x$ 入栈,此时 $x$ 会成为新的栈顶。\\ 
$s.pop()$ 将栈顶出栈,此时原本栈顶下的那个元素会成为新的栈顶。\\ 
$s.size()$ 返回当前栈内元素的个数。 \\ 
由于栈只需要维护栈顶,所以我们无法实现在非栈顶的位置的插入和删除操作。
\end{frame}
\begin{frame}
\frametitle{模拟栈 -- I}
B3614 【模板】栈\\
实现栈的方式非常多,在数据结构的课程中通常使用指针的方式实现,但是在算法竞赛中,出于速度和内存安全性的考虑,我们通常使用静态数组模拟栈,之后介绍的几个数据结构同样会采用静态数组模拟。 \\ 
可以使用一个数组 $s[1..n]$ 来表示栈,用一个变量 $top$ 表示当前的栈顶,初始时 $top = 0$,如果入栈就让 $top += 1$,再为 $s[top]$ 赋值,出栈就让 $top -= 1$。所以栈的元素数量等于 $top$,判断栈是否为空就看 $top$ 是否为 $0$ 。
\end{frame}
\begin{frame}[fragile]
\frametitle{模拟栈 -- II}
\begin{onlyenv}
\begin{minted}[fontsize = \scriptsize]{cpp}
void solve(void) {
    top = 0;
    int n; std::cin >> n;
        while(n--) {
            std::string op; std::cin >> op;
            if(op == "push") {
                u64 x; std::cin >> x;
                s[++top] = x;
            } else if(op == "pop") {
                if(top > 0) {
                    top--;
                } else {
                    std::cout << "Empty" << std::endl;
                }
            } else if(op == "query") {
                if(top > 0) std::cout << s[top] << std::endl;
                else std::cout << "Anguei!" << std::endl;
            } else {
                std::cout << top << std::endl;
            }
        }
}
\end{minted}
\end{onlyenv}
\end{frame}
\begin{frame}[fragile]
\frametitle{STL中的栈 -- I}
在STL库中,我们可以使用 $stack$ 容器。\\
$stack$ 有以下迭代器: \\ 
$s.size()$ 返回栈中的元素个数 \\
$s.empty()$ 返回栈是否为空 \\
$s.top()$ 返回栈顶元素 \\
$s.push(x)$ 将 $x$ 入栈 \\
$s.pop()$ 将栈顶出栈
\end{frame}
\begin{frame}[fragile]
\frametitle{STL中的栈 -- II}
所以对于这道题,我们可以这样写:\\
\begin{onlyenv}
\begin{minted}[fontsize = \scriptsize]{cpp}
void solve(void) {
    std::stack<u64> s;
    int n; std::cin >> n;
    while(n--) {
        std::string op; std::cin >> op;
        if(op == "push") {
            u64 x; cin >> x; s.push(x);
        }
        else if(op == "pop") {
            if(!s.empty()) s.pop();
            else cout << "Empty" << endl;
        } else if(op == "query") {
            if(!s.empty()) cout << s.top() << endl;
            else cout << "Anguei!" << endl;
        } else cout << s.size() << endl;
    }
}
\end{minted}
\end{onlyenv}
\end{frame}
\section{队列}
\begin{frame}
\frametitle{介绍}
\fdf{队列}(Queue) 是一种\fdf{先进先出}的数据结构。\\
队列通常能实现以下操作:\\
$q.front()$ 访问\fdf{队首}元素\\
$q.push(x)$ 将 $x$ 入队,此时 $x$ 会成为新的\fdf{队首}。\\ 
$q.pop()$ 将\fdf{队尾}出队,此时原本队尾前的那个元素会成为新的队尾。\\ 
$q.size()$ 返回当前队内元素的个数。 \\ 
\end{frame}
\begin{frame}
\frametitle{模拟队列 -- I}
B3616 【模板】队列 \\ 
与模拟栈类似,我们同样是使用一个静态数组 $q[1..n]$ 来表示队列,不同的是我们需要维护两个指针 $head$ 和 $tail$ 分别表示队首和队尾。开始时 $head$ 和 $tail$ 都指向 $0$ ,如果入队就让 $tail += 1$,再为 $q[tail]$ 赋值;如果出队就让 $head += 1$ ,如果 $head$ 和 $tail$ 重合说明此时队列为空,队列内的元素个数应为 $head - tail$ ,队首元素应为 $q[head + 1]$ 。
\end{frame}
\begin{frame}[fragile]
\frametitle{模拟队列 -- II}
\begin{onlyenv}
\begin{minted}[fontsize = \scriptsize]{cpp}
void solve(void) {
    int Q; cin >> Q;
    while(Q--) {
        string op; cin >> op;
        if(op == "1") {
            i64 x; cin >> x;
            q[++tail] = x;
        } else if(op == "2") {
            if(head == tail) std::cout << "ERR_CANNOT_POP" << std::endl;
            else head ++;
        } else if(op == "3") {
            if(head == tail) std::cout << "ERR_CANNOT_QUERY" << std::endl;
            else std::cout << q[head+1] << std::endl;
        } else {
            std::cout << tail - head << std::endl;
        }
    }
}
\end{minted}
\end{onlyenv}
\end{frame}
\begin{frame}
\frametitle{STL中的队列 -- I}
在STL库中,我们可以使用 $queue$ 容器。\\
$queue$ 有以下迭代器: \\ 
$q.size()$ 返回队列中的元素个数 \\
$q.empty()$ 返回队列是否为空 \\
$q.front()$ 返回队首元素 \\
$q.push(x)$ 将 $x$ 入队 \\
$q.pop()$ 将队尾出队
\end{frame}
\begin{frame}[fragile]
\frametitle{STL中的队列 -- II}
对于这道题,我们可以这样写:
\begin{onlyenv}
\begin{minted}[fontsize = \scriptsize]{cpp}
void solve(void) {
    std::queue<int> q;
    int Q; std::cin >> Q;
    while(Q--) {
        int a; cin >> a;
        if(a == 1) {
            int x; std::cin >> x;
            q.push(x);
        } else if(a == 2) {
            if(q.size()) q.pop();
            else std::cout << "ERR_CANNOT_POP" << std::endl;
        } else if(a == 3) {
            if(!q.empty()) std::cout << q.front() << std::endl;
            else std::cout << "ERR_CANNOT_QUERY" << std::endl;
        } else {
            std::cout << q.size() << std::endl;
        }
    }
}
\end{minted}
\end{onlyenv}
\end{frame}
\section{链表}
\begin{frame}
\frametitle{介绍 -- I}
链表有非常多种形式,但是在算法竞赛主要的作用是存储\fdf{图} (Graph),因此本节我们只介绍单向链表和双向链表。\\
\fdf{链表} (Linked\_List) 是一种可以方便的在元素中间进行插入和删除操作的数据结构,链表由\fdf{节点} (Node) 和连接节点的指针组成,每个节点存储本节点的元素和它指向的下一个链表的指针(双链表还会存储它的上一个节点的指针)。由于节点并不是连续的,因此无法像数组一样进行 $O(1)$ 的随机访问,只能通过 $O(n)$ 的遍历寻找节点。
\end{frame}
\begin{frame}
\frametitle{介绍 -- II}
单链表的节点存储键值和它指向的下一个节点的指针(next)。\\ 
单链表支持两种操作:插入和删除。\\ 
对于插入操作,假设要在 $Node_1$ 和 $Node_2$ 中间插入 $Node_3$,那么过程是这样的:创建 $Node_3$,原本 $Node_1$ 指向 $Node_2$,随着 $Node_3$ 的加入,指向关系变成了 $Node_1 \rightarrow Node_3 \rightarrow Node_2$,插入操作就结束了。\\ 
对于删除操作,假设要删除 $Node_1$ 和 $Node_2$ 之间的 $Node_3$,那么只需要更改它们的指向关系,由$Node_1 \rightarrow Node_3 \rightarrow Node_2$ 变为 $Node_1 \rightarrow Node_2$ ,由于没有节点指向 $Node_3$ ,作为一个无法访问到的节点,我们就默认为它被删除了。
\end{frame}
\begin{frame}
\frametitle{介绍 -- III}
双链表的节点存储键值、它指向的下一个节点的指针、指向它的上一个节点的指针,称为\fdf{前驱}(pre)和\fdf{后继}(next)。\\ 
双链表同样支持插入和删除的操作。\\ 
对于插入操作,假设还是在 $Node_1$ 和 $Node_2$ 之间插入 $Node_3$,那么过程是这样的:创建 $Node_3$ ,之后 $Node_1$ 的后继指向 $Node_3$ ,$Node_3$ 的前驱指向 $Node_1$ 、后继指向 $Node_2$, $Node_2$ 的前驱指向 $Node_3$。\\
对于删除操作,假设删除 $Node_1$ 和 $Node_2$ 之间的 $Node_3$,那么将 $Node_1$ 的后继指向 $Node_2$, $Node_2$ 的前驱指向 $Node_1$ 即可。
\end{frame}
\begin{frame}[fragile]
\frametitle{模拟链表 -- I}
B3631 单向链表\\
数组模拟的链表,也叫静态链表。由于这道题过于简单,因此我们只需要一个 $next$ 数组,用下标表示键值,用对应的元素表示指向的指针即可:\\
\begin{onlyenv}
\begin{minted}[fontsize = \scriptsize]{cpp}
int next[N];
void solve(void) {
    next[1] = 0;
    int Q; std::cin >> Q;
    while(Q--) {
        int op, x, y; std::cin >> op;
        if(op == 1) {
            std::cin >> x >> y;
            next[y] = next[x];
            next[x] = y;
        } else if(op == 2) {
            std::cin >> x;
            std::cout << next[x] << std::endl;
        } else {
            std::cin >> x;
            if(next[x] == 0) continue;
            else next[x] = next[next[x]];
        }
    }
}
\end{minted}
\end{onlyenv}
\end{frame}
\begin{frame}[fragile]
\frametitle{模拟链表 -- II}
B4324 【模板】双向链表 \\ 
本题作为双向链表的模版题有些特殊,因为一开始所有的数已经练成了一个链表。因此进行操作一和操作二的时候,我们要在 $Node_y$ 的左边或者右边插入 $Node_x$ 的前提是如果 $Node_x$ 存在于链表上,我们要先把它从链表中移出来,也就是做一次删除操作。所以我们可以用结构题来存储节点,每个节点有前驱、后继、是否被删除三个属性,用下标来保存键值。
\begin{onlyenv}
\begin{minted}[fontsize = \scriptsize]{cpp}
struct Node {
    int pre, next;
    bool deleted;
}l[N];
void solve(void) {
    int n, m; std::cin >> n >> m;
    for(int i = 1; i <= n; i++) {
        l[i].pre = i - 1;
        l[i].next = i + 1;
    }
    l[1].pre = 0, l[n].next = 0; l[0].next = 1;
\end{minted}
\end{onlyenv}
\end{frame}
\begin{frame}[fragile]
\frametitle{模拟链表 -- III}
\begin{onlyenv}
\begin{minted}[fontsize = \scriptsize]{cpp}
    while(m--) {
        int op, x, y; std::cin >> op;
        if(op == 1) {
            std::cin >> x >> y;
            if(x == y) continue;
            if(!l[x].deleted) {
                l[l[x].pre].next = l[x].next;
                l[l[x].next].pre = l[x].pre;
                l[x].deleted = true;
            }
            int p = l[y].pre;
            l[p].next = x;l[x].pre = p;
            l[x].next = y;l[y].pre = x;
            l[x].deleted = false;
        } else if(op == 2) {
            std::cin >> x >> y;
            if(x == y) continue;
            if(!l[x].deleted) {
                l[l[x].pre].next = l[x].next;
                l[l[x].next].pre = l[x].pre;
                l[x].deleted = true;
\end{minted}
\end{onlyenv}
\end{frame}
\begin{frame}[fragile]
\frametitle{模拟链表 -- IV}
\begin{onlyenv}
\begin{minted}[fontsize = \scriptsize]{cpp}
            }
            int p = l[y].next;
            l[x].next = p;l[p].pre = x;
            l[y].next = x;l[x].pre = y;
            l[x].deleted = false;
        } else {
            std::cin >> x;
            if(!l[x].deleted) {
                l[l[x].pre].next = l[x].next;
                l[l[x].next].pre = l[x].pre;
                l[x].deleted = true;
            }
        }
    }int cur = l[0].next;
    if(cur == 0) {
        std::cout << "Empty!" << std::endl;
        return;
    }
    while(cur) {
        std::cout << cur << " "; cur = l[cur].next;
    } std::cout << std::endl;
\end{minted}
\end{onlyenv}
\end{frame}
\begin{frame}
\frametitle{STL中的链表 -- I}
在STL中提供了双链表容器 $list$ \footnote{其实还提供了单链表容器 $forward\_list$ ,其使用方法和list基本一致,感兴趣的同学可以自行查阅资料。},支持以下操作:\\ 
$L.begin(), L.end()$ $list$ 的首、尾迭代器。 \\ 
$L.push_back(x)$ 在链表末端插入一个元素 $x$ 。\\ 
$L.front(), L.back()$ 返回链表的首、尾迭代器。 \\ 
$L.splice(x, A, y)$ 从链表 $A$ 中将第 $x$ 个节点转移到链表 $L$ 的 $y$ 号节点后。
$L.erase(x)$ 将链表 $L$ 中的 $x$ 号节点删除。
\end{frame}
\begin{frame}[fragile]
\frametitle{STL中的链表 -- II}
对于上面的题目,我们可以使用 $List$ 解决。
\begin{onlyenv}
\begin{minted}[fontsize = \scriptsize]{cpp}
void solve(void) {
    int n, m; std::cin >> n >> m;
    std::list<int> L;
    std::vector<bool> deleted(n + 1, false);
    std::vector<std::list<int>::iterator> pos(n + 1);
    for(int i = 1; i <= n; i++) {
        L.push_back(i);
        auto it = L.end();
        --it; pos[i] = it;
    }
    while(m--) {
        int op, x, y;
        std::cin >> op;
        if(op == 1) {
            std::cin >> x >> y;
            if(x == y) continue;
            L.splice(pos[y], L, pos[x]);
\end{minted}
\end{onlyenv}
\end{frame}
\begin{frame}[fragile]
\frametitle{STL中的链表 -- III}
\begin{onlyenv}
\begin{minted}[fontsize = \scriptsize]{cpp}
        } else if(op == 2) {
            std::cin >> x >> y;
            if(x == y) continue;
            auto it = pos[y]; it++;
            L.splice(it, L, pos[x]);
        } else {
            std::cin >> x;
            if(!deleted[x]) {
                pos[x] = L.erase(pos[x]);
                deleted[x] = true;
            }
        }
    }
    if(L.empty()) {
        std::cout << "Empty!" << std::endl;
        return;
    }
    for(auto i : L) {
        std::cout << i << " ";
    }
}
\end{minted}
\end{onlyenv}
\end{frame}
\section{习题}
\begin{frame}
\frametitle{习题}
线性表是数据结构的基础,之后我们学习的很多内容都需要依靠今天学到的内容才能实现,大家一定要勤加练习。\\
P3156	【深基15.例1】询问学号\\ 
P3613	【深基15.例2】寄包柜	\\
P1449	后缀表达式	\\
P1996	约瑟夫问题 \\
P1160	队列安排	\\
P1540	[NOIP 2010 提高组] 机器翻译	\\
P2058	[NOIP 2016 普及组] 海港	\\
P1241	括号序列	\\	
P4387	【深基15.习9】验证栈序列	\\
P2234	[HNOI2002] 营业额统计 \\
\end{frame}
\end{document}