\documentclass{beamer}

%---------- 主题设置 ----------
\usetheme{Berlin}               % 可选主题:Dresden, CambridgeUS, Malmoe…
\usecolortheme{orchid}          % 可选配色:beaver, orchid, seagull…
\setbeamertemplate{navigation symbols}{} % 隐藏导航图标

%---------- 中文字体配置 ----------
\usepackage[UTF8, fontset=mac]{ctex}

%---------- 常用宏包 ----------
\usepackage{graphicx}
\usepackage{listings}
\usepackage{tikz}
\usepackage{minted}
\usepackage{changepage}
\usepackage{graphbox}
\usepackage{twemojis}
\usepackage{relsize}
\newcommand{\myemoji}[1]{\raisebox{-.2ex}{\twemoji[height=1.25\fontcharht\font`X]{#1}}}
%---------- 自定义命令 ----------
% C++ 符号宏,依赖 hyperref 和 relsize
% 来自 GitHub 上的仓库 tcbrindle/wg21papertemplate 中的 ./common.tex 文件
\newcommand{\Cpp}{\texorpdfstring{C\kern-0.05em\protect\raisebox{.35ex}{\textsmaller[2]{+\kern-0.05em+}}}{C++}}
% 举例:让 \fdf 变成加粗显示
\newcommand{\fdf}[1]{\alert{\textbf{#1}}}
% 如果你只是想要一个空行,也可以用 \vspace{.5\baselineskip} 或者 \medskip
\newcommand{\emptyline}{\vspace{.5\baselineskip}}


% 用于数学公式中的自定义命令
\newcommand{\nequiv}{\not\equiv} % 不同余
\newcommand{\lcm}{\operatorname{lcm}} % 最小公倍数
\newcommand{\lpd}{\operatorname{lpd}} % 最小素因数

\title{最短路}
\author{dbywsc}
\date{2025/8}
\begin{document}
\frame{\titlepage}
\begin{frame}
	\frametitle{目录}
	\tableofcontents
\end{frame}
\section{介绍}
\begin{frame}
最短路是非常常见的图论问题,分为\fdf{单源最短路径}和\fdf{多源最短路径}\\ 
本节中的一些约定:\\ 
$n$ 为图中点的数量, $m$ 为图中边的数量。\\ 
$s$ 为最短路的源点。 \\ 
$w(u, v)$ 为边 $(u, v)$ 的边权。\\ 
$dis(u, v)$ 为点 $u$ 到 $v$ 最短路距离。
常用的最短路算法有三种,本节将一一介绍。
\end{frame}
\section{Floyd}
\begin{frame}
Floyd是用来求任意两个节点直接的最短路算法。\\ 
适用于有向图、无向图、负权图,但是前提是最短路必须存在\footnote{最短路存在的前提是不能有\fdf{负环}}。\\ 
Floyd的实现使用了动态规划的思想,设 $f_{k, x, y}$ 为只允许经过点 $1$ 到 $k$ (其中不一定包括 $x$ 和 $y$),节点 $x$ 到 $y$ 之间到最短路长度。考虑如何状态转移,可以发现一共有两种情况,第一种情况是 $f_{k, x, y}$ 直接由 $f_{k - 1, x, y}$ 转移过来,也就是不走 $k$ 这个点;第二种情况是 $f_{k, x, y}$ 从 $f_{k - 1, x, k} + f_{k - 1, k, y}$ 转移过来,也就是先从 $x$ 走到 $k$ ,再从 $k$ 走到 $y$ ,我们每次在两者之间取一个最小值就好了。因此状态转移方程为:
$$f_{k, x, y} = \min(f_{k - 1, x, y}, f_{k - 1, x, k} + f_{k - 1, k, y})$$
接下来考虑边界,由于要求最小值,所以要初始化为最大值,但是当 $x = y$ 时,显然应该将距离初始化为 $0$ ;当存在一条权为 $w$ 的边 $x \rightarrow y$ 时,要让 $f_{0, x, y} = w$。
\end{frame}
\begin{frame}[fragile]
\begin{onlyenv}
\begin{minted}[fontsize = \scriptsize]{cpp}
void solve(void) {
    memset(f, 0x3f, sizeof(f));
    int n, m; std::cin >> n >> m;
    for(int i = 1; i <= m; i++) {
        int u, v, w; std::cin >> u >> v >> w;
        G[u][v] = G[v][u] = (G[u][v] ? std::min(G[u][v], w) : w);
        f[0][u][v] = f[0][v][u] = std::min(f[0][u][v], w);
    }
    for(int k = 0; k <= n; k++)
    for(int i = 1; i <= n; i++) f[k][i][i] = 0;
    for(int k = 1; k <= n; k++) 
        for(int x = 1; x <= n; x++) 
            for(int y = 1; y <= n; y++) 
                f[k][x][y] = std::min(f[k - 1][x][y], f[k - 1][x][k] + f[k - 1][k][y]);
    for(int i = 1; i <= n; i++) {
        for(int j = 1; j <= n; j++)
            std::cout << f[n][i][j] << " ";
        std::cout << "\n";
    }
}
\end{minted}
\end{onlyenv}
floyd算法通常搭配邻接矩阵实现。上面的代码中,时空复杂度均为 $O(n^3)$
\end{frame}
\begin{frame}[fragile]
考虑用滚动数组的方式优化,可以发现 $f$ 的第一维对结果没有影响,因此可以省去第一维:
\begin{onlyenv}
\begin{minted}[fontsize = \scriptsize]{cpp}
for(int k = 1; k <= n; k++) {
    for(int x = 1; x <= n; x++) {
        for(int y = 1; y <= n; y++) {
            f[x][y] = std::min(f[x][y], f[x][k] + f[k][y]);
        }
    }
}
\end{minted}
\end{onlyenv}
此时的时间复杂度为 $O(n^3)$ ,空间复杂度为 $O(n^2)$ 。
\end{frame}
\begin{frame}
B3647 【模板】Floyd\\ 
由于floyd的时空复杂度都很高,所以它只能处理点数比较小的图。此外,floyd常被用来求多源最短路、找最小环、封闭传包等问题。
\end{frame}
\section{Dijkstra}
\begin{frame}
Dijkstra是由Dijkstra\footnote{E.W.Dijkstra}发明的,一种用于求解\fdf{非负权图}上单源最短路径的算法。\\ 
Dijkstra的流程如下:\\ 
将图上所有的点分为两个点集:已经确定最短路的集合 $S$ 和还没有确定的集合 $T$ ,初始时,所有的点都属于 $T$ 。 \\ 
初始化 $dis(s) = 0$ ,其他为极大值。\\ 
然后重复地从 $T$ 中选取一个最短路长度最小的节点,移到 $S$ 中,并且对被刚刚加入 $S$ 的节点的所有出边进行\fdf{松弛}操作,直到 $T$ 为空。\\ 
\end{frame}
\begin{frame}
松弛操作,即对于一条边 $(u, v)$ ,执行下面的式子:$dis(v) = \min(dis(v), dis(u) + w(u, v))$ 。\\
这样做的含义即尝试用 $S \rightarrow u \rightarrow v$ (保证 $S \rightarrow u$ 已经是最短路)这条路径更新 $v$ 点的最短路,如果存在更优的方案,就进行更新。
\end{frame}
\begin{frame}
可以发现,Dijkstra的实现瓶颈主要在于“寻找最短路中长度最小的点”。朴素的做法是在点集 $T$ 中通过枚举的方式暴力地找最小点,时间复杂度为 $O(n^2)$ ,这个效率我们无法接受\footnote{事实上,朴素的Dijkstra在稠密图中表现更优,但是我们一般不考虑},因此在本节中,我们不会介绍朴素做法的实现。 \\ 
考虑优化,可以发现,如果使用堆(即C++中的优先队列)来存放点集 $T$ ,可以保证堆顶一定最短路长度最小的点,此时我们将时间复杂度优化到了 $O(\log n)$ 。\\
而“对 $S$ 的所有出边进行松弛操作”的时间复杂度为 $O(m)$ ,因此堆优化后的 $Dijkstra$ 时间复杂度为 $O(m \log n)$ 。
\end{frame}
\begin{frame}[fragile]
P4779 【模板】单源最短路径(标准版)\\ 
\begin{onlyenv}
\begin{minted}[fontsize = \scriptsize]{cpp}
void dijkstra(void) {
    memset(dis, 0x3f, sizeof(dis));
    dis[s] = 0;
    std::priority_queue<PII, std::vector<PII>, std::greater<PII> > q;
    q.push({0, s}); //由于pair默认按照first进行排序,所以我们先dis再v
    while(q.size()) {
        auto u = q.top().y; q.pop();
        if(st[u]) continue;
        st[u] = 1;
        for(auto ed : G[u]) {
            int v = ed.x, w = ed.y;
            if(dis[v] > dis[u] + w) {
                dis[v] = dis[u] + w;
                q.push({dis[v], v});
            }
        }
    }
}
\end{minted}
\end{onlyenv}
\end{frame}
\begin{frame}
可以发现,dijkstra本质上是一个特殊的BFS。
直到现在,dijkstra仍然是最优秀的单源最短路算法,所以在处理最短路问题时,绝大部分问题都应该用dijkstra解决。\\ 
但是,dijkstra无法处理负权图,此时应该使用别的算法。
\end{frame}
\section{SPFA}
\begin{frame}
SPFA是Bellman–Ford算法的一种实现\footnote{严格意义上来说,SPFA是朴素Bellman-Ford的队列优化,SPFA即Shortest Path Faster Algorithm},在CNOI/CPC中,选手们更热衷于称其为“SPFA” 。\\
SPFA同样是依赖于松弛操作的一种最短路算法,特点是能够处理负权图和判断是否存在最短路。\\ 
SPFA的流程如下:
将 $s$ 入队。 \\ 
当队列不为空的时候,重复地对队首的点的出边进行松弛操作,同时将枚举到的新点入队。 \\ 
直到队列为空。 \\
最坏情况下,SPFA的时间复杂度是O(nm)。\\ 
与Dijkstra不同的是,SPFA能够处理负权边,这就导致遇到负环时,SPFA会陷入死循环中,这个时候我们可以记录一下,如果SPFA已经松弛了 $m$ 条边后还在循环中,就可以确定遇到了负环,直接退出即可。
\end{frame}
\begin{frame}[fragile]
P3371 【模板】单源最短路径(弱化版)\\ 
\begin{onlyenv}
\begin{minted}[fontsize = \scriptsize]{cpp}
void spfa(void) {
    memset(dis, 0x3f, sizeof(dis));
    memset(cnt, 0, sizeof(cnt));
    std::queue<int> q;
    st[s] = 1; q.push(s); dis[s] = 0;
    while(q.size()) {
        auto u = q.front(); q.pop();
        st[u] = 0;
        for(auto ed : G[u]) {
            if(dis[v] > dis[u] + w) {
                dis[v] = dis[u] + w;
                cnt[v] = cnt[u] + 1;
                if(cnt[v] >= n) return;
                if(!st[v]) {
                    st[v] = 1; q.push(v);
                }
            }
        }
    }
}
\end{minted}
\end{onlyenv}
可以发现,SPFA其实也是一个特殊的BFS。
\end{frame}
\begin{frame}
\begin{large}
关于SPFA\\
\end{large}
\begin{small}
它死了。\footnote{出自NOI 2018 Day1} \\ 
\end{small}
\\
SPFA算法本身有许多潜在的问题,很容易被卡掉。因此在现代算法竞赛中,大部分非负权图的最短路问题都会卡SPFA。因此,在能够使用Dijkstra时务必要使用Dijkstra。当遇到负权图、存在负环的图、或者差分约束等Dijkstra无法解决的问题时,再考虑用SPFA解决。
\end{frame}
\section{习题}
\begin{frame}
由于最短路问题有非常多的扩展,因此本节的习题除了最短路的基础问题外,还有一些拓展性的问题。\\ 
P2910	[USACO08OPEN] Clear And Present Danger S\\ 
P3905	道路重建\\ 
P1144	最短路计数\\ 
P2136 拉近距离 \\
P2446	[SDOI2010] 大陆争霸 \\
P6175	无向图的最小环问题 \\
P5837	[USACO19DEC] Milk Pumping G\\ 
P3385	【模板】负环 \\ 
P5960	【模板】差分约束
\end{frame}
\end{document}