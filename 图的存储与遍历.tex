\documentclass{beamer}

%---------- 主题设置 ----------
\usetheme{Berlin}               % 可选主题:Dresden, CambridgeUS, Malmoe…
\usecolortheme{orchid}          % 可选配色:beaver, orchid, seagull…
\setbeamertemplate{navigation symbols}{} % 隐藏导航图标

%---------- 中文字体配置 ----------
\usepackage[UTF8, fontset=mac]{ctex}

%---------- 常用宏包 ----------
\usepackage{graphicx}
\usepackage{listings}
\usepackage{tikz}
\usepackage{minted}
\usepackage{changepage}
\usepackage{graphbox}
\usepackage{twemojis}
\usepackage{relsize}
\newcommand{\myemoji}[1]{\raisebox{-.2ex}{\twemoji[height=1.25\fontcharht\font`X]{#1}}}
%---------- 自定义命令 ----------
% C++ 符号宏,依赖 hyperref 和 relsize
% 来自 GitHub 上的仓库 tcbrindle/wg21papertemplate 中的 ./common.tex 文件
\newcommand{\Cpp}{\texorpdfstring{C\kern-0.05em\protect\raisebox{.35ex}{\textsmaller[2]{+\kern-0.05em+}}}{C++}}
% 举例:让 \fdf 变成加粗显示
\newcommand{\fdf}[1]{\alert{\textbf{#1}}}
% 如果你只是想要一个空行,也可以用 \vspace{.5\baselineskip} 或者 \medskip
\newcommand{\emptyline}{\vspace{.5\baselineskip}}


% 用于数学公式中的自定义命令
\newcommand{\nequiv}{\not\equiv} % 不同余
\newcommand{\lcm}{\operatorname{lcm}} % 最小公倍数
\newcommand{\lpd}{\operatorname{lpd}} % 最小素因数

\title{图的存储与遍历}
\author{dbywsc}
\date{2025/8}
\begin{document}
\frame{\titlepage}
\begin{frame}
	\frametitle{目录}
	\tableofcontents
\end{frame}
\section{图的相关概念}
\begin{frame}
\fdf{图论} (Graph theory) 是数学的一个分支,图是图论的主要研究对象\footnote{在计算机科学,或者在算法竞赛中,图论的一些概念或者符号表示与数学中的图论不完全一样。}。\fdf{图} (Graph) 是由若干给定的顶点及连接两顶点的边构成的图形,这种图形通常用来描述某些事物之间的某种特定关系。顶点用于代表事物,连接两顶点的边则用于表示两个事物间具有这种关系。
\end{frame}
\begin{frame}
图是一个二元组 $G = (V(G), E(G))$ 。其中 $V(G)$ 是非空集,称为\fdf{点集} (vertex set) ,对于 $V$ 中点每个元素,我们称其为\fdf{顶点}(vertex) 或者\fdf{节点} (node) ,简称为 \fdf{点}; $E(G)$ 为各点之间\fdf{边} (edge) 的集合,称为\fdf{边集} (edge set) 。 \\ 
常用 $G = (V, E)$ 来表示图。 \\ 
图分为\fdf{无向图} (undirected graph) ,\fdf{有向图} (directed graph) 和\fdf{混合图} (mixed graph) 。\\ 
若 $G$ 为无向图,则 $E$ 中的每一个元素为一个无序二元组 $(u, v)$ ,称作\fdf{无向边} (undirected edge) ,其中 $u, v \in V$ 。设 $e = (u, v)$ , 则称 $u$ 和 $v$ 为 $e$ 的\fdf{端点} (endpoint) 。\\
若 $G$ 为有向图,则 $E$ 中的每一个元素为一个有序二元组 $(u, v)$ ,称作\fdf{有向边} (directed edge) 或者 \fdf{弧} (arc) ,有时也写作 $u \rightarrow v$ 。 设 $e = u \rightarrow v$ ,则 $u$ 为 $e$ 的\fdf{起点} (tail) , $v$ 为 $e$ 的\fdf{终点} (head) 。起点和终点称为 $e$ 的端点。并称 $u$ 为 $e$ 的直接前驱, $v$ 为 $e$ 的直接后继。
\end{frame}
\begin{frame}
若 $G$ 为混合图,则其中既有有向边,又有无向边。\\
若 $G$ 的每一条边 $e_k = (u_k, v_k)$ 都被赋予一个数作为这个边的\fdf{权},则称 $G$ 为\fdf{带权图}。如果这个图的权均为正实数,则称图为\fdf{正权图}。\\
图 $G$ 的点数 $| V(G) |$ 称为图的\fdf{阶} 。 \\ 
与一个点 $v$ 关联的边的条数称为\fdf{度}(degree) ,记做 $d(v)$ 。特别地,对于边 $(u, v)$ ,则每条这样的边要对 $d(v)$ 产生	$2$ 的贡献。\\ 
\end{frame}
\begin{frame}
\fdf{途径}(walk):是连接一连串顶点的点的序列,可以为有限或者无限长度。形式化地说,一条有限途径 $w$ 是一个边的序列 $e_1, e_2, ..., e_k$ ,是的存在一个顶点序列 $v_0, v_1, v_2, ... v_k$ ,满足 $e = (v_{i - 1}, v_i)$ ,其中, $i \in [1, k]$ 。这样的途径可以简写为 $v_0 \rightarrow v_1 \rightarrow v_2 \rightarrow ... \rightarrow v_k$ 。通常来说,边的数量 $k$ 被称为这条边的 \fdf{长度} (如果边是带权的,长度则为权重之和)。 \\
\fdf{迹} (trail):对于一条途径 $w$ , 若 $e_1, e_2, ..., e_k$ 两两互不相同,则称 $w$ 是一条迹。 \\ 
\fdf{简单路径} (simple path) (又称\fdf{路径} (path) ):对于一条迹 $w$ ,若其连接点的序列中点两两不同,则称	 $w$ 是一条路径。 \\
\fdf{回路} (circuit):对于一条迹 $w$,若 $v_0 = v_k$,则称 $w$ 是一条回路。 \\ 
\fdf{环/圈}(cycle)(又称\fdf{简单回路/简单环} (simple curcuit)):对于一条回路 $w$,若 $v_0 = v_k$ 是点序列中唯一重复出现的点对,则称	 $w$ 是一个环。 
\end{frame}
\begin{frame}
\fdf{自环}(loop):对 $E$ 中对边 $e = (u, v)$ ,若 $u = v$,则 $e$ 被称作自环。\\ 
\fdf{重边}(multiple edge):若 $E$ 中存在两个完全相同的元素(边) $e_1, e_2$ ,则它们被称作(一组)重边。 \\ 
\fdf{简单图} (simple graph):若一张图中没有重边和自环,则它被称为简单图,否则,称它为 \fdf{多重图} (multi graph) 。
\end{frame}
\section{图的存储方式}
\begin{frame}
通常有四种存图方式,分别为:直接存边,邻接矩阵,邻接表,链式前向星\footnote{链式前向星是OI中的说法,它更正式的表达叫邻接链表}。 \\ 
本节中,用 $n$ 指代图的点数,用 $m$ 指代图的边数,用 $d^+(u)$ 指代点 $u$ 的出度,即以 $u$ 为出发点的边数。
\end{frame}
\begin{frame}[fragile]
直接存边即定义一个结构体数组,结构体中存储边的属性,比如起点、终点、权值等。\\ 
\begin{onlyenv}
\begin{minted}[fontsize = \scriptsize]{cpp}
struct Edge {
    int u, v, w;
};
std::vector<Edge> edges(m);
\end{minted}
\end{onlyenv}
复杂度: \\
查询是否存在某条边: $O(m)$ 。\\ 
遍历一个点的所有出边: $O(m)$ 。\\ 
遍历整张图: $O(nm)$ 。 \\ 
空间复杂度: $O(m)$ 。 \\ 
直接存边的效率比较低下,一般不会使用这种做法。但是在 $Kruskal$ 算法中,由于需要对所有的边按边权排序,因此需要使用直接存边。在某些题目中,如果涉及多次建图的操作,需要重复使用这些边,也可以先将所有的边存下来。
\end{frame}
\begin{frame}
邻接矩阵即使用一个二维数组 $adj$ 来存边,使用 $adj[u][v] = 1$ 来标记存在一条边 $e = (u, v)$,如果为 $0$ 则表示不存在,如果是带权边,则可以使用 $adj[u][v] = w$ 表示存在一条边 $u \rightarrow v$ ,且边权为 $w$ 。 
对于无向边,我们直接令 $adj[u][v] = 1$ 并且 $adj[v][u] = 1$,即存储两条两个方向的边。 \\
复杂度:\\
查询是否存在某条边: $O(1)$ 。\\ 
遍历一个点的所有出边: $O(n)$ 。\\ 
遍历整张图: $O(n^2)$ 。 \\ 
空间复杂度: $O(n^2)$ 。 \\ 
邻接矩阵无法处理需要存储重边的情况。它的优势是可以 $O(1)$ 地查询一条边是否存在,但是总体来说,它的时空复杂度都很高。只能在小规模内使用,并且一般用于处理稠密图\footnote{边的条数 $| E |$ 远远小于$| v |^2$ 的图称为\fdf{稀疏图},如果两者接近,则称为\fdf{稠密图}。}。在应用上,$Floyd$ 算法需要依赖邻接矩阵。
\end{frame}
\begin{frame}[fragile]
邻接表的实现是使用一个支持动态增加元素的数据结构构成的数组(例如 $std::vector<int> adj[n + 1]$ )来存边,其中 $adj[u]$ 存储了点 $u$ 的所有出边。\\ 
对于无向边,显然可以通过存储两条有向边的方式实现。\\
对于带权图,我们可以额外定义一个表示边的数据结构,存储终点和权值,可以使用 $std::pair<int, int>$ 来实现。\\ 
\end{frame}
\begin{frame}[fragile]
存储无权图:
\begin{onlyenv}
\begin{minted}[fontsize = \scriptsize]{cpp}
std::vector<int> adj[N];
int main(void) {
    for(int i = 1; i <= m; i++) {
        int u, v; std::cin >> u >> v;
        adj[u].push_back(v);
        adj[v].push_back(u);
    }
}
\end{minted}
\end{onlyenv}
存储带权图:
\begin{onlyenv}
\begin{minted}[fontsize = \scriptsize]{cpp}
typedef std::pair<int, int> PII;
std::vector<PII> adj[N];
int main(void) {
    for(int i = 1; i <= m; i++) {
        int u, v, w; std::cin >> u >> v >> w;
        G[u].emplace_back(v, w);
    }
}
\end{minted}
\end{onlyenv}
\end{frame}
\begin{frame}
复杂度:\\
查询是否存在 $u$ 到 $v$ 的边: $O(d^+(u))$ (如果实现做了排序,可以使用二分查找,时间复杂度为 $O(\log(d^+(u))$ )。\\ 
遍历点 $u$ 点所有出边: $O(d^+(u))$ 。\\ 
遍历整张图:$O(n + m)$ 。 \\
空间复杂度: $O(m)$ 。 \\ 
相对来说,领接表的时空复杂度都很优秀,可以处理大部分问题,接下来的所有图论算法我们都将使用领接表演示。
\end{frame}
\begin{frame}
链式前向星(邻接链表)本质上是使用链表实现的领接表。在 $OI$ 中被广泛使用。 \\ 
复杂度:\\ 
查询是否存在 $u$ 到 $v$ 的边: $O(d^+(u))$ 。\\ 
遍历点 $u$ 点所有出边: $O(d^+(u))$ 。\\ 
遍历整张图:$O(n + m)$ 。 \\
空间复杂度: $O(m)$ 。 \\ 
由于使用了静态链表实现,常数上来说速度稍快于邻接表。\\
同样能够处理绝大部分问题,由于存边时带了编号,有时会非常有用,比如在网络流算法中就会使用这种写法。同时,在 $Java$ 等常数较慢的语言中是最佳的选择。\footnote{对这种方法感兴趣的同学可以查阅OI-wiki}
\end{frame}
\begin{frame}[fragile]
\begin{onlyenv}
\begin{minted}[fontsize = \scriptsize]{cpp}
// head[u] 和 cnt 的初始值都为 -1
void add(int u, int v) {
  nxt[++cnt] = head[u];  // 当前边的后继
  head[u] = cnt;         // 起点 u 的第一条边
  to[cnt] = v;           // 当前边的终点
}

// 遍历 u 的出边
for (int i = head[u]; ~i; i = nxt[i]) {  // ~i 表示 i != -1
  int v = to[i];
}
\end{minted}
\end{onlyenv}
\end{frame}
\section{图的遍历}
\begin{frame}
之前我们已经接触过了 $DFS$ 和 $BFS$ 算法,事实上,这两个算法本来就是一种图上遍历的算法。
\end{frame}
\begin{frame}[fragile]
与搜索中的 $DFS$ 不同,图上 $DFS$ 是有对应模版的:
\begin{onlyenv}
\begin{minted}[fontsize = \scriptsize]{cpp}
void dfs(int u) {
    vis[u] = true;
    for(auto v : G[u]) {
        if(!vis[v]) dfs(v);
    }
}
\end{minted}
\end{onlyenv}
$DFS$ 的时间复杂度为 $O(m + n)$ ,空间复杂度为 $O(n)$ ,同时,由于 $DFS$ 递归依赖栈空间,因此,栈空间的空间复杂度也为 $O(n)$ 。 \\
对于一张图,对其进行 $DFS$ 遍历,得到的节点的顺序称为 \fdf{DFS序} 。 \\ 
对于联通图来说, DFS序通常不唯一。 
\end{frame}
\begin{frame}[fragile]
$BFS$ 是图论中最重要、最基础的算法之一。事实上,之前我们就已经给出过图上 $BFS$ 的模版。
\begin{onlyenv}
\begin{minted}[fontsize = \scriptsize]{cpp}
void bfs(int start) {
    queue<int> q;
    q.push(start); vis[start] = 1;
    while(q.size()) {
        auto u = q.front(); q.pop();
        for(auto v : G[u]) {
            if(!vis[v]) {
                vis[v] = 1; q.push(v);
            }
        }
    }
}
\end{minted}
\end{onlyenv}
时间复杂度为 $O(n + m)$ ,空间复杂度为 $O(n)$ 。
相对于 $DFS$ 序,$BFS$ 过程中访问到的节点的顺序叫做 $BFS$ 序,$BFS$ 序通常不唯一。
\end{frame}
\begin{frame}
现在,来练习一道模版题:P5318 【深基18.例3】查找文献
\end{frame}
\section{拓扑排序}
\begin{frame}
\fdf{拓扑排序}(Topological sorting) 要解决的问题是如何给一个\fdf{有向无环图}(简称DAG)排序。\\ 
与传统意义上的排序不同,这里的排序指的是针对节点之间的依赖关系做一个线性的排序。比如要想学习算法竞赛,你需要学习编程语言、高等数学、离散数学、数据结构等等,它们之间的依赖关系是,先学会了高等数学,才能学习离散数学,先学习了编程语言,才能学习数据结构,而离散数学和数据结构之间又存在关联......如果我们把每个课程看作一个点,把它们之间的关系作为边,那我们刚刚就相当于是在做拓扑排序。\\ 
因此我们可以说,在一个 DAG 中,我们将途中的顶点以线性的方式进行排序,使得对任何一个顶点 $u$ 到 $v$ 的有向边 $(u, v)$ ,都可以有 $u$ 在 $v$ 的前面。 \\
还有给定一个 DAG ,如果从 $i$ 到 $j$ 有边,则认为 $j$ 依赖于 $i$ 。 如果 $i$ 到 $j$ 有路径($i$ 可达 $j$),则称 $j$ 间接依赖于 $i$ 。 \\ 
\fdf{拓扑排序的目标是将所有节点排序,使得排在前面的节点不能依赖于排在后面的节点。}
\end{frame}
\begin{frame}
构造拓扑排序的步骤是:\\ 
1.从图中的选择一个入度为 $0$ \footnote{如果存在一条有向边 $u \rightarrow v$ ,那么这条边为 $u$ 增加了 $1$ 的出度,为 $v$ 增加了 $1$ 的入度。} 的点。 \\
2.输出并删除这个点,连带删除这个点所有的出边。	 \\ 
拓扑排序的实现有 $DFS$ 和 $BFS$ 两种,我们在这里只介绍 $BFS$ 实现。 \\ 
$BFS$ 实现的拓扑排序又叫 $Kahn$ 算法,初始状态下,集合 $S$ 装着所有入度为 $0$ 的点,$L$ 是一个空列表。
每次从 $S$ 中取出一个点(可以随便去) $u$ 放入 $L$,然后将 $u$ 和 $u$ 的所有出边删除,此时可能会有新的点入度变成了 $0$ ,将它们也放入 $S$ 中。 不断重复以上过程,直到 $S$ 为空,如果此时图中还存在边,说明存在环路,这张图并非 DAG ,因此拓扑序不存在。否则,此时依次输出 $L$ 中的点,就是要求的拓扑序。  \\ 
时间复杂度:由于本质上是一个 $BFS$ ,因此复杂度为 $O(E + V)$ 。
\end{frame}
\begin{frame}[fragile]
\begin{onlyenv}
\begin{minted}[fontsize = \scriptsize]{cpp}
bool toposort(void) {
    std::vector<int> L;
    std::queue<int> s;
    for(int i = 1; i <= n; i++)
        if(in[i] == 0) s.push(i);
    while(s.size()) {
        auto u = s.front(); s.pop();
        L.push_back(u);
        for(auto v : G[u]) {
            if(--in[v] == 0) s.push(v);
        }
    }
    if(L.size() == n) {
        for(auto i : L) std::cout << i << " ";
        return true;
    }
    return false;
}
\end{minted}
\end{onlyenv}
\end{frame}
\begin{frame}
B3644 【模板】拓扑排序 / 家谱树s
\end{frame}
\section{习题}
\begin{frame}
P3916   图的遍历\\ 
P2853   [USACO06DEC] Cow Picnic S \\ 
P1347   排序 \\ 
P1983   [NOIP 2013 普及组] 车站分级 \\
P1038   [NOIP 2003 提高组] 神经网络 \\
P1807   最长路\\
P4017 最大食物链计数
\end{frame}
\end{document}